\documentclass[12pt]{article}

\usepackage[utf8]{inputenc}
\usepackage[ngerman]{babel}
\usepackage{microtype}

\title{Tractatus logico-philosophicus}
\author{Ludwig Wittgenstein\\[0.3cm]
\small
Übersetzung vom Spanischen (wieder) ins Deutsche: Francisco Escobar}
\date{Bearbeitet am \today}

\begin{document}

\maketitle

\section{Vorwort}

Möglicherweise wird dieses Buch nur derjenige verstehen, der bereits wenn nicht
die in diesem Buch dargestellten, dann zumindest ähnliche Gedanken konzipiert
hat. Es ist also kein Handbuch. Sein Ziel würde erreicht werden, wenn es dem,
der es liest und begreift, Vergnügen bereiten würde. Das Buch handelt von den
philosophischen Problemen und zeigt, glaube ich, dass deren Formulierung auf
dem Unverständnis der Logik unserer Sprache aufruht. Den gesamten Sinn des
Buchs kann man in folgenden Worten zusammenfassen: Was überhaupt gesagt werden
kann, kann klar gesagt werden, und über das, worüber man nicht reden kann,
muss man schweigen. Das Buch versucht dann, dem Nachdenken eine Grenze zu
setzen, oder eigentlich nicht dem Nachdenken, sondern dem Ausdruck der
Gedanken. Denn um dem Nachdenken eine Grenze zu setzen, müsste man in der Lage
sein, sich die zwei Seiten dieser Grenze vorzustellen (im Endeffekt müsste man
daran denken können, was undenkbar ist).

So kann die Grenze nur der Sprache gezogen werden und das, was jenseits der
Grenze liegt, ist einfach absurd.

Inwiefern meine Bemühungen denen anderer Philosophen zustimmen, will ich nicht
beurteilen. Was ich hier geschrieben habe, strebt sicherlich keinerlei
Neuheit an, weswegen ich keine Quellen angebe. Es ist mir egal, ob das, was
ich mir überlegt habe, von anderen überlegt worden ist.

Ich möchte nur erwähnen, dass sich ein großer Teil der Anregung zu meinen
eigenen Gedanken den großartigen Werken Freges und meines Freundes 
Bertrand Russell verdanken.

Wenn diese Arbeit irgendeinen Wert hat, dann hat sie ihn in zweierlei
Hinsicht. Erstens, weil in ihr Gedanken ausgedrückt werden. Je besser
diese Gedanken ausgedrückt sind, je näher dem Schwarzen getroffen wird, desto
höher ist dieser Wert. In dieser Hinsicht bin ich mir dessen bewusst, weniger
als das Mögliche getan zu haben, aus dem einfachen Grund, dass zur Vollendung
dieser Aufgabe meine Kraft nicht ausreicht. Ich hoffe, es wird andere geben,
die es besser tun.

Die Wahrheit der hier vermittelten Gedanken erscheint mir dagegen unantastbar
und endgültig. Ich bin also der Meinung, im Wesentlichen die Probleme 
gelöst zu haben. Wenn ich mich nicht irre, besteht zweitens der Wert dieses
Werks darin, gezeigt zu haben, wie wenig man durch die Lösung dieser Probleme
zu Stande gebracht hat.

\section{Tractatus}

\newenvironment{en}{\begin{enumerate}}{\end{enumerate}}

\begin{itemize}
    \item Die Welt ist alles, was der Fall ist.
    \item Die Welt ist die Gesamtheit der Tatsachen, nicht der Dinge.
    \item Die Welt ist durch die Tatsachen bestimmt und davon, dass diese
    \emph{all} die Tatsachen sind.
    \item Denn die Gesamtheit der Tatsachen bestimmt, was der Fall bzw. nicht
    der Fall ist.
    \item Die Tatsachen im logischen Raum sind die Welt.
    \item Die Welt zerfällt in Tatsachen.
    \item Etwas kann entweder der Fall sein oder nicht der Fall sein, während
    alles Übrige gleich bleibt.
    \item Was der Fall ist, die Tatsache, ist das effektive Sichergeben von
    Sachverhalten.
    \item Der Sachverhalt ist eine Verbindung von Gegenständen (Dingen).
    \item Vom Sachverhalt Teil sein zu können, ist für die Sache wesentlich.
    \item In der Logik ist nichts zufällig. Wenn die Sache geschehen
    \emph{kann,} muss die Möglichkeit des Sachverhalts bereits in der Sache
    selbst vorhanden sein.
\end{itemize}
\end{document}

